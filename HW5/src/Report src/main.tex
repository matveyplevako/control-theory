\documentclass[a4paper,11pt]{article}

\usepackage[T1]{fontenc}
\usepackage[utf8]{inputenc}
\usepackage{graphicx}
\usepackage{xcolor}

\renewcommand\familydefault{\sfdefault}
\usepackage{tgheros}

\usepackage{amsmath,amssymb,amsthm,textcomp}
\usepackage{enumerate}
\usepackage{multicol}
\usepackage{tikz}
\usepackage{pdfpages}
\usepackage{hyperref}


\graphicspath{ {.} }
\usepackage{geometry}
\geometry{left=25mm,right=25mm,%
bindingoffset=0mm, top=20mm,bottom=20mm}



\usepackage{tabularx,lipsum,environ,amsmath,amssymb}

\makeatletter
\newcommand{\problemtitle}[1]{\gdef\@problemtitle{#1}}% Store problem title
\newcommand{\problemquestion}[1]{\gdef\@problemquestion{#1}}% Store problem question
\newcommand{\problemsolution}[1]{\gdef\@problemsolution{#1}}% Store problem input
\NewEnviron{problem}{
  \problemtitle{}\problemquestion{}\problemsolution{}% Default input is empty
  \BODY% Parse input
  \par\addvspace{.5\baselineskip}
  \noindent
  \begin{tabularx}{\textwidth}{@{\hspace{\parindent}} l X c}
    \multicolumn{2}{@{\hspace{\parindent}}l}{\@problemtitle} \\% Title
    \textbf{Description:} & \@problemquestion \\% Question
        \textbf{Solution:} & \@problemsolution % Input
  \end{tabularx}
  \par\addvspace{.5\baselineskip}
}
\makeatother



\linespread{1.3}

\newcommand{\linia}{\rule{\linewidth}{0.5pt}}

% custom theorems if needed
\newtheoremstyle{mytheor}
    {1ex}{1ex}{\normalfont}{0pt}{\scshape}{.}{1ex}
    {{\thmname{#1 }}{\thmnumber{#2}}{\thmnote{ (#3)}}}

\theoremstyle{mytheor}
\newtheorem{defi}{Definition}

% my own titles
\makeatletter
\renewcommand{\maketitle}{
\begin{center}
\vspace{2ex}
{\huge \textsc{\@title}}
\vspace{1ex}
\\
\linia\\
\@author \hspace{100ex} m.plevako@innopolis.university \hspace{100ex} BS18-02 \hspace{100ex} Variant (c)

\vspace{4ex}
\end{center}
}
\makeatother
%%%

% custom footers and headers
\usepackage{fancyhdr}
\pagestyle{fancy}
\lhead{}
\chead{}
\rhead{}
\lfoot{Assignment \textnumero{} 5}
\cfoot{}
\rfoot{Page \thepage}
\renewcommand{\headrulewidth}{0pt}
\renewcommand{\footrulewidth}{0pt}
% 

% code listing settings
\usepackage{listings}
\lstset{
    language=Python,
    basicstyle=\ttfamily\small,
    aboveskip={1.0\baselineskip},
    belowskip={1.0\baselineskip},
    columns=fixed,
    extendedchars=true,
    breaklines=true,
    tabsize=4,
    prebreak=\raisebox{0ex}[0ex][0ex]{\ensuremath{\hookleftarrow}},
    frame=lines,
    showtabs=false,
    showspaces=false,
    showstringspaces=false,
    keywordstyle=\color[rgb]{0.627,0.126,0.941},
    commentstyle=\color[rgb]{0.133,0.545,0.133},
    stringstyle=\color[rgb]{01,0,0},
    numbers=left,
    numberstyle=\small,
    stepnumber=1,
    numbersep=10pt,
    captionpos=t,
    escapeinside={\%*}{*)}
}

%%%----------%%%----------%%%----------%%%----------%%%

\begin{document}

\title{HW \textnumero{} 5}

\author{Matvey Plevako}

\maketitle


\section*{Problem}

% \includepdf[pages=-]{HW3_task1.pdf}


\section*{Problem 1}
Consider classical benchmark system in control theory - inverted pendulum on a cart (Figure 1). It is nonlinear under-actuated system that has the following dynamics.
  $$(M + m)\ddot{x} - ml \cos(\theta) \ddot{\theta}+ ml \sin(\theta) \dot{\theta}^2 = F$$
$$- \cos(\theta)\ddot{x} + l\ddot{\theta}- g \sin(\theta) = 0$$
\begin{center}
    where $g = 9.81$ is gravitational acceleration.
    $$(c)M = 3.6,m = 3.6,l = 1.01$$
The system dynamics can be written in state space form:
$$\dot{z} = f (z) + g(z)u$$
$$y = h(z) = \begin{bmatrix}x & \theta\end{bmatrix}^T$$

where $z =  \begin{bmatrix}x & \theta & \dot{x} & \dot{\theta}\end{bmatrix}^T$
is the state vector of the system, y is the output vector. The dynamics of the system around unstable equilibrium of the pendulum $(\bar{z} = \begin{bmatrix} 0 & 0 & 0 & 0 \end{bmatrix}^T)$ can be described by a linear system that is obtained from linearization of the nonlinear dynamics around $ \bar{z} $.

$$\delta \dot{z} = A\delta z + B\delta u$$
$$\delta y = C\delta z$$
    $$\includegraphics[width=10cm, height=10cm]{scheme.png}$$
    Figure 1: A schematic drawing of the inverted pendulum on a cart. The rod is considered massless. The mass of the cart and the point mass at the end of the rod are denoted by M and m. The rod has a length l.
\end{center}



\section*{part A}
\begin{problem}
  \problemquestion{prove that it is possible to design state observer of the linearized system}
  \problemsolution{
System is observable, if matrix $S = \begin{bmatrix} C \\ CA \\ CA^2 \\ CA^3\end{bmatrix}$ has rank 4
$$C = \begin{bmatrix} 1 & 0 & 0 & 0 \\ 0 & 1 & 0 & 0 \end{bmatrix}$$

$$CA = \begin{bmatrix} 0 & 0 & 1 & 0 \\ 0 & 0 & 0 & 1 \end{bmatrix}$$

$$CA^2 = \begin{bmatrix} 0 & \frac{mg}{M} & 0 & 0 \\ 0 & \frac{g(m+M)}{lM} & 0 & 0 \end{bmatrix}$$

$$CA^3 = \begin{bmatrix} 0 & 0 & 0 & \frac{mg}{M} \\ 0 & 0 & 0 & \frac{g(m+M)}{lM}\end{bmatrix}$$
$$S = \begin{bmatrix} 1 & 0 & 0 & 0 \\ 0 & 1 & 0 & 0 \\ 0 & 0 & 1 & 0 \\ 0 & 0 & 0 & 1 \\ 0 & \frac{mg}{M} & 0 & 0 \\ 0 & \frac{g(m+M)}{lM} & 0 & 0 \\ 0 & 0 & 0 & \frac{mg}{M} \\ 0 & 0 & 0 & \frac{g(m+M)}{lM} \end{bmatrix}$$
We can see the Identity matrix 4x4 at the upper part of the S matrix => its rank is 4
  }
\end{problem}


\section*{part B}
\begin{problem}
  \problemquestion{for open loop state observer, is the error dynamics stable?}
  \problemsolution{Open-loop state observer has a form: $\hat{\dot{z}} = A\hat{z} + Bu$
  Error dynamics:
    
$\epsilon = \hat{z} - z$, $\dot{z} = Az + Bu$ , $\dot{\epsilon} = A \epsilon $

Thus, open loop state observe is stable, when A is negative definite, which is not the case, becaue:

$Det (\begin{bmatrix} -\lambda & 0 & 1 \\ 0 & -\lambda & 0 & 1 \\ 0 & \frac{mg}{M} & -\lambda & 0 \\ 0 & \frac{g(M+m)}{lM} & 0 & -\lambda \end{bmatrix}) = -\lambda(-\lambda^3 + \lambda(\frac{g(M+m)}{lM})) = \lambda^2(\lambda^2 - \frac{g(M+m)}{lM})$ if $\lambda = 0 $ or $\lambda = \pm \sqrt{\frac{g(M+m)}{lM}}$ $\lambda = \pm 4,4052174287$}
\end{problem}

\section*{part C}
\begin{problem}
  \problemquestion{design Luenberger observer for linearized system using both pole placement and LQR methods}
  \problemsolution{Luenberger observer has form: 
$$\hat{\dot{z}}_{k+1} = A\hat{z_k} + Bu_k + L (y_k - \hat{y_k})$$
$$\hat{y_k} = C\hat{z_k} + Du_k$$
In the given case, $D = 0$
Pole placement method:

it is usually used in case: $A - BL \prec 0 $, 

now the system $A - LC \prec  0 $ is given, if it is transposed: $A^T-C^TL^T \prec 0 $ 
$$L^T = poles(A^T, C^T, eigVals)$$

LQR method (possible, because, C is a part of Identity matrix):

$$L^T = lqr(A^T, C^T, Q, R)$$
}
\end{problem}

\begin{lstlisting}[label={list:first}]
import numpy as np
import matplotlib.pyplot as plt
import scipy.signal as sig
from scipy.integrate import odeint
import scipy.linalg as lin

g = 9.81
M = 3.6
m = 3.6
l = 1.01

eig = [-1.1, -1.2, -1.3, -1.4]

A = np.array([[0, 0, 1, 0], [0, 0, 0, 1], [0, g*m/M, 0, 0], [0, g*(M+m)/l/M, 0, 0]])
B = np.array([0, 0, 1/M, 1/l/M]).reshape(1, -1).T
C = np.array([[1, 0, 0, 0], [0, 1, 0, 0]])
#  pole placement method
pole = sig.place_poles(A.T, C.T, eig)
L_pole = pole.gain_matrix.T

# lqr method
# Q, R - random, but appropriate
Q = np.array([[1, 0, 0, 0], 
              [0, 1, 0, 0], 
              [0, 0, 1, 0], 
              [0, 0, 0, 1]])

R = np.array([[4, 1], [1, 4]]) 

S = lin.solve_continuous_are(A.T, C.T, Q, R)
L_lqr = np.array(np.linalg.inv(R)).dot(C).dot(S).T

pole = sig.place_poles(A, B, eig)
P = -pole.gain_matrix

def usual(x, t, u):
    n = np.dot(A, x) + np.dot(B, u)
    return  n

def observer(x_hat, t, u, x):
    return np.dot(A, x_hat) + np.dot(B, u) + np.dot(L_lqr, C).dot(x - x_hat)

dt = 1/10000
T = 20
time = np.linspace(0, T, dt**(-1))

x = [np.array([0.01, 0.001, 0.01, 0.01])]
x_hat = [np.array([0.02, 0.002, 0.02, 0.02])/4]

for i in range(1, len(time)):
#     Use odeint between two dots,
#     but u is fixed between two poins
#     P controller u = Px
    local_time = np.linspace(time[i-1], time[i])
    u = np.dot(P, x[-1])

    x_dot = odeint(usual, x[-1], local_time, args=tuple([u]))
    x.append(x_dot[-1])
    
    x_hat_dot = odeint(observer, x_hat[-1], local_time, args=tuple([u, x[-1]]))
    x_hat.append(x_hat_dot[-1])

def plot_sim(x, x_hat, time):
    x = np.array(x)
    y = np.dot(C, x.T)
    x_hat = np.array(x_hat)
    y_hat = np.dot(C, x_hat.T)

    plt.figure(figsize=(8, 8))
    plt.title("Inverted pendulum system simulation")
    plt.plot(time, y[0], label="init velocity")
    plt.plot(time, y[1], label="init coordinate")

    plt.plot(time, y_hat[0], label="obs velocity")
    plt.plot(time, y_hat[1], label="obs coordinate")
    plt.grid()
    plt.legend()

plot_sim(x, x_hat, time)
\end{lstlisting}

\includegraphics[width=10cm, height=9cm]{C.png}

\section*{part D}
\begin{problem}
  \problemquestion{design state feedback controller for linearized system}
  \problemsolution{}
\end{problem}
\begin{lstlisting}[label={list:second}]
res_pole = sig.place_poles(A, B, eig)
K = res_pole.gain_matrix

# visualization
def control(x, t):
    return np.dot(A - np.dot(B, K), x)

time = np.linspace(0, 20, 1000)   
x0 = x[0]
res = odeint(control, x0, time).T

fig = plt.figure(figsize=(8, 8))
plt.title("Stabilisation of the system")
plt.xlabel("time")
plt.plot(time, res[0], "r-", label="x")
plt.plot(time, res[1], "b-", label="$\theta$")
plt.plot(time, res[2], "k-", label="$\dot{x}$")
plt.plot(time, res[3], "g-", label="$\dot{\theta}$")
plt.grid()
plt.legend(shadow=True)
plt.show()
\end{lstlisting}{}

\includegraphics[width=10cm, height=9cm]{C.png}


\section*{part E}
\begin{problem}
  \problemquestion{Simulate nonlinear system with Luenberger observer and state feedback controller that uses estimated states ($u = K \hat{x}$). Make sure that the system is stabilized for various initial conditions around $\bar{z}$..}
  \problemsolution{}
\end{problem}

\begin{lstlisting}
import numpy as np
import matplotlib.pyplot as plt
import scipy.signal as sig
from scipy.integrate import odeint
import scipy.linalg as lin

g = 9.81
M = 3.6
m = 3.6
l = 1.01

eig = [-1.1, -1.2, -1.3, -1.4]

A = np.array([[0, 0, 1, 0], [0, 0, 0, 1], [0, g*m/M, 0, 0], [0, g*(M+m)/l/M, 0, 0]])
B = np.array([0, 0, 1/M, 1/l/M]).reshape(1, -1).T
C = np.array([[1, 0, 0, 0], [0, 1, 0, 0]])
#  pole placement method
pole = sig.place_poles(A.T, C.T, eig)
L_pole = pole.gain_matrix.T

# lqr method
# Q, R - random, but appropriate
Q = np.array([[1, 0, 0, 0], 
              [0, 1, 0, 0], 
              [0, 0, 1, 0], 
              [0, 0, 0, 1]])

R = np.array([[4, 1], [1, 4]]) 

S = lin.solve_continuous_are(A.T, C.T, Q, R)
L_lqr = np.array(np.linalg.inv(R)).dot(C).dot(S).T


def usual(x, t, u):
    n = np.dot(A, x) + np.dot(B, u)
    return  n


def observer(x_hat, t, u, x):
#     TRY WITH BOTH: L_lqr and L_pole
    return np.dot(A, x_hat) + np.dot(B, u) + np.dot(L_lqr, C).dot(x - x_hat)

dt = 1/10000
T = 20
time = np.linspace(0, T, dt**(-1))

x = [np.array([0.01, 0.001, 0.01, 0.01])]
x_hat = [np.array([0.02, 0.002, 0.02, 0.02])/4]

for i in range(1, len(time)):
#     Use odeint between two dots,
#     but u is fixed between two poins
#     P controller u = Px
    local_time = np.linspace(time[i-1], time[i])
    u = np.dot(P, x[-1])

    x_dot = odeint(usual, x[-1], local_time, args=tuple([u]))
    x.append(x_dot[-1])
    
    x_hat_dot = odeint(observer, x_hat[-1], local_time, args=tuple([u, x[-1]]))
    x_hat.append(x_hat_dot[-1])

plot_sim(x, x_hat, time)

\end{lstlisting}

\includegraphics[width=10cm, height=9cm]{C.png}


\section*{part F}
\begin{problem}
  \problemquestion{Add white gaussian noise to the output ($\delta y = C\delta z + v$).}
  \problemsolution{}
\end{problem}


\begin{lstlisting}
def usual(x, t, u):
    n = np.dot(A, x) + np.dot(B, u)
    return  n


def observer(x_hat, t, u, dy):
    return np.dot(A, x_hat) + np.dot(B, u) + np.dot(L_lqr, dy - np.dot(C, x_hat))

dt = 1/10000
T = 20
time = np.linspace(0, T, dt**(-1))

x = [np.array([0.01, 0.001, 0.01, 0.01])]
x_hat = [np.array([0.02, 0.002, 0.02, 0.02])/4]


for i in range(1, len(time)):
#     BUT u is fixed between two poins
#     P controller u = P
    local_time = np.linspace(time[i-1], time[i])
    u = np.dot(P, x[-1])

    x_dot = odeint(usual, x[-1], local_time, args=tuple([u]))
    x.append(x_dot[-1])
    
    dy = np.dot(C, x[-1]) 
    dy += np.random.random(2) * 0.0005
    x_hat_dot = odeint(observer, x_hat[-1], local_time, args=tuple([u, dy]))
    x_hat.append(x_hat_dot[-1])

plot_sim(x, x_hat, time)
plt.show()
\end{lstlisting}

\includegraphics[width=10cm, height=10cm]{F.png}



\section*{part G}
\begin{problem}
  \problemquestion{Add white gaussian noise to the dynamics ($\delta \dot{z} = A\delta z + B\delta u + w$). What happens to the state estimation and control system?}
\end{problem}

\begin{lstlisting}
def usual(x, t, u):
    n = np.dot(A, x) + np.dot(B, u) + np.random.random(4) * 0.00005
    return  n


def observer(x_hat, t, u, dy):
    return np.dot(A, x_hat) + np.dot(B, u) + np.dot(L_lqr, dy - np.dot(C, x_hat))

dt = 1/10000
T = 20
time = np.linspace(0, T, dt**(-1))

x = [np.array([0.01, 0.001, 0.01, 0.01])]
x_hat = [np.array([0.02, 0.002, 0.02, 0.02])/4]


for i in range(1, len(time)):
#     use odeint between two dots,
#     but u is fixed between two poins
#     P controller u = Px
    local_time = np.linspace(time[i-1], time[i])
    u = np.dot(P, x[-1])

    x_dot = odeint(usual, x[-1], local_time, args=tuple([u]))
    x.append(x_dot[-1])
    
    dy = np.dot(C, x[-1]) 
    dy += np.random.random(2) * 0.0005
    x_hat_dot = odeint(observer, x_hat[-1], local_time, args=tuple([u, dy]))
    x_hat.append(x_hat_dot[-1])
    
plot_sim(x, x_hat, time)
plt.show()

\end{lstlisting}

\includegraphics[width=10cm, height=10cm]{G.png}


\section*{part H}
\begin{problem}
  \problemquestion{implement Kalman Filter}
  \problemsolution{Prediction
  $$X^-_k = A_{k-1}X_{k-1}+B_kU_k$$
  $$P^-_k = A_{k-1}P_{k-1}A^T_{k-1}+Q_{k-1}$$
  Update
  $$V_k = Y_k-H_kX^-_k$$
  $$S_k = H_kP^-_kH^T_k+Q_{k-1}$$
  $$K_k = P^-_k H^T_kS^{-1}_k$$
  $$X_k = X^-_k + K_kV_k$$
  $$P_k = P^-_k + K_kS_kK^T_k$$
  }
\end{problem}

\begin{lstlisting}
class KalmanFilter(object):
    def __init__(self, F = None, B = None, H = None, Q = None, R = None, P = None, x0 = None):
        self.n = F.shape[1]
        self.m = H.shape[1]

        self.F = F
        self.H = H
        self.B = 0 if B is None else B
        self.Q = np.eye(self.n) if Q is None else Q
        self.R = np.eye(self.n) if R is None else R
        self.P = np.eye(self.n) if P is None else P
        self.x = np.zeros((self.n, 1)) if x0 is None else x0

    def predict(self, u = 0):
        self.x = np.dot(self.F, self.x) + np.dot(self.B, u)
        self.P = np.dot(np.dot(self.F, self.P), self.F.T) + self.Q
        return self.x

    def update(self, z):
        y = z - np.dot(self.H, self.x)
        S = self.R + np.dot(self.H, np.dot(self.P, self.H.T))
        K = np.dot(np.dot(self.P, self.H.T), np.linalg.inv(S))
        self.x = self.x + np.dot(K, y)
        I = np.eye(self.n)
        self.P = np.dot(np.dot(I - np.dot(K, self.H), self.P),  
                        (I - np.dot(K, self.H)).T) + np.dot(np.dot(K, self.R), K.T)
\end{lstlisting}{}


\section*{part I}
\begin{problem}
  \problemquestion{generate some data and show that your implementation of KF is correct}
\end{problem}

\begin{lstlisting}
dt = 1.0/60
F = np.array([[1, dt, 0], [0, 1, dt], [0, 0, 1]])
H = np.array([1, 0, 0]).reshape(1, 3)
Q = np.array([[0.05, 0.05, 0.0], [0.05, 0.05, 0.0], [0.0, 0.0, 0.0]])
R = np.array([0.5]).reshape(1, 1)

x = np.linspace(-10, 10, 100)
measurements = - (x**2 + 2*x - 2)  + np.random.normal(0, 2, 100)

kf = KalmanFilter(F = F, H = H, Q = Q, R = R)
predictions = []

for z in measurements:
    predictions.append(np.dot(H,  kf.predict())[0])
    kf.update(z)


plt.plot(range(len(measurements)), measurements, label = 'Measurements')
plt.plot(range(len(predictions)), np.array(predictions), label = 'Kalman Filter Prediction')
plt.legend()
plt.show()
\end{lstlisting}{}

\includegraphics[width=15cm, height=10cm]{I.png}

\section*{part J}
\begin{problem}
  \problemquestion{using KF function implement LQG controller}
\end{problem}

\begin{lstlisting}
M = 1000;
m1 = 100;
m2 = 100;
l1 = 20;
l2 = 10;
g = 9.80;
x0= [ 5 ; 0 ; 0.1 ; 0 ;0.2 ; 0 ; 0 ; 0 ; 0 ; 0 ; 0 ; 0]
A= [0 1 0 0 0 0 ; 0 0 -(m1*g/M) 0 -(m2*g/M) 0 ; 0 0 0 1 0 0 ; 0 0 -(g*(M+m1))/(M*l1) 0 -(m2*g)/(M*l1) 0 ;0 0 0 0 0 1; 0 0 -(m1*g)/(M*l2) 0 -(g*(M+m2))/(M*l2) 0]
B= [ 0; 1/M ;0; 1/(M*l1) ;0 ;1/(M*l2)];
C = [1 0 0 0 0 0];
D = 0;
Q=[1 0 0 0 0 0;0 1 0 0 0 0; 0 0 100 0 0 0; 0 0 0 1000 0 0; 0 0 0 0 150 0; 0 0 0 0 0 1500]
R=0.0001
K = lqr(A,B,Q,R);
sys_1 = ss(A,[B B],C,[zeros(1,1) zeros(1,1)]);
vd = 0.3;
vn = 1;
sen = [1];
known = [1];
[~,L,~] = kalman(sys_1,vd,vn,[],sen,known)
Ac = [A-B*K B*K;zeros(size(A)) A-L*C];
Bc = zeros(12,1);
Cc = [C zeros(size(C))];
sys_cl_lqg = ss(Ac,Bc,Cc,D );

t = 0:0.01:100;
F = zeros(size(t));
[Y,~,X] = lsim(sys_cl_lqg,F,t,x0);
figure
plot(t,Y(:,1),'b');
u = zeros(size(t));
for i = 1:size(X,1)
u(i) = K * (X(i,1:6))';
end
Xhat = X(:,1) - X(:,6);
figure(2);
hold on
plot(t,Xhat)

plot(t,X(:,1),'r')
legend('X__hat','X')
hold off



x0 =

    5.0000
         0
    0.1000
         0
    0.2000
         0
         0
         0
         0
         0
         0
         0


A =

         0    1.0000         0         0         0         0
         0         0   -0.9800         0   -0.9800         0
         0         0         0    1.0000         0         0
         0         0   -0.5390         0   -0.0490         0
         0         0         0         0         0    1.0000
         0         0   -0.0980         0   -1.0780         0


Q =

           1           0           0           0           0           0
           0           1           0           0           0           0
           0           0         100           0           0           0
           0           0           0        1000           0           0
           0           0           0           0         150           0
           0           0           0           0           0        1500


R =

   1.0000e-04


L =

    0.0303
    0.0005
    0.0000
    0.0000
    0.0001
    0.0000

\end{lstlisting}{}

\includegraphics[width=15cm, height=10cm]{J.png}


\end{document}
